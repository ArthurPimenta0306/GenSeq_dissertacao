
\chapter{Revisão Bibliográfica}
\label{subsection:structure_pred}

O objetivo deste capítulo é situar o presente trabalho no contexto das contribuições existentes na área de pesquisa.
Para isso, analisamos diversas publicações da literatura que estão alinhadas com os objetivos deste estudo.


\section{Trabalhos relacionados}
% Tratamentos Atuais para Hemofilia B  
O tratamento atual para hemofilia B baseia-se na reposição do FIX por meio de proteínas recombinantes ou derivadas do plasma humano. 
A introdução de fatores de coagulação recombinantes na prática clínica representou um marco no manejo da doença, 
proporcionando maior segurança e eficácia em comparação com derivados do plasma, 
reduzindo significativamente os riscos de transmissão de patógenos (\cite{COHEN1995675}).  
No entanto, as terapias baseadas na reposição de FIX ainda apresentam desafios substanciais.
Devido à curta meia-vida do FIX na circulação, 
pacientes geralmente necessitam de infusões frequentes para manter níveis terapêuticos adequados,
o que aumenta os custos e a complexidade do tratamento (\cite{Mancuso}). 
Além disso, os pacientes podem desenvolver inibidores, 
que são anticorpos neutralizantes dirigidos contra o FIX,
comprometendo a eficácia do tratamento (\cite{Mancuso2}).
Avanços recentes, como o desenvolvimento de variantes do FIX de longa duração como o rFIX-Fc, 
estenderam a meia-vida da proteína, reduzindo a frequência de infusões (\cite{Massimo}). 
Apesar disso, essas terapias continuam a ser custosas e, em alguns casos,
limitadas pela resposta imunológica do paciente.  


% Design de Proteínas  
O design de proteínas surge como uma alternativa com potencial de desenvolver tratamentos mais eficientes. 
Trata-se de um campo emergente que busca criar ou modificar proteínas para otimizar suas funções biológicas, 
estabilidade estrutural ou características terapêuticas. 
Os avanços nessa área têm sido impulsionados pelo aprimoramento de tecnologias de modelagem molecular
e pelo uso de ferramentas computacionais especializadas.
O software Rosetta (\cite{Rosetta}) por exemplo,
foi uma das primeiras ferramentas amplamente utilizadas para predição de estruturas proteicas.
Seu funcionamento baseia-se na minimização da energia livre de uma proteína, 
modelando interações moleculares fundamentais como interações de Van der Waals,
forças eletrostáticas, ligações de hidrogênio e efeitos hidrofóbicos.
A predição estrutural utilizando o Rosetta fundamenta-se no princípio de que a conformação nativa de uma proteína 
corresponde a conformação de menor energia livre. 
O processo é iterativo e inicia-se com a geração de modelos estruturais iniciais, 
frequentemente por meio da montagem de fragmentos, 
em que pequenos segmentos estruturais de proteínas conhecidas são recombinados para gerar conformações aproximadas da proteína alvo.
Para cada modelo gerado é estimado a energia livre da conformação.
A partir dessas estimativas, o Rosetta aplica métodos estocásticos, 
como Monte Carlo Simulated Annealing, para explorar diferentes conformações e minimizar a função de energia,
conduzindo a um refinamento progressivo da estrutura modelada. 
Os modelos que apresentam menores valores são considerados candidatos mais prováveis para representar a estrutura nativa
da proteína em estudo.


\cite{Alphafold2} introduziu o \textit{AlphaFold}, um modelo que
representou um avanço significativo na predição estrutural de proteínas,
ao introduzir uma abordagem baseada em \textit{deep learning} que superou 
métodos computacionais tradicionais.
O problema da predição estrutural pode ser formalmente descrito como a minimização da função de energia livre associada 
à conformação da proteína. Diferentemente dos métodos tradicionais que exploram o espaço conformacional por meio de amostragem estocástica,
o \textit{AlphaFold} aprende diretamente um mapa de distâncias e ângulos que minimiza essa função de energia.  
O modelo utiliza redes neurais treinadas em bancos de dados contendo estruturas proteicas determinadas experimentalmente, 
como as obtidas por cristalografia de raios X, espectroscopia por ressonância magnética nuclear (RMN) e criomicroscopia eletrônica. 
A base teórica do \textit{AlphaFold} fundamenta-se na 
modelagem das interações residuais a partir de alinhamentos múltiplos de sequências (MSA, Multiple Sequence Alignments) 
e do aprendizado das representações espaciais dos resíduos por meio de camadas de atenção baseadas em grafos.
O modelo recebe como entrada uma sequência de aminoácidos e gera um conjunto de representações internas (\textit{embeddings})
que capturam informações sobre a proximidade e orientação relativa dos resíduos na estrutura tridimensional da proteína.
O modelo foi amplamente validado por meio de sua participação na 
edição 14 do \textit{CASP (Critical Assessment of Structure Prediction)}, 
uma avaliação que compara métodos de predição de estrutura proteica com dados experimentais inéditos. 
No CASP14, o \textit{AlphaFold} demonstrou um desempenho quase equivalente ao de técnicas experimentais.
Esse avanço revolucionou o campo da biologia estrutural,
reduzindo a dependência exclusiva de métodos experimentais e abrindo novas possibilidades para a engenharia de proteínas
 e a compreensão de doenças associadas a dobramentos proteicos anômalos.


\cite{ESMFold} apresentaram o \textit{ESMFold}, 
um modelo baseado em \textit{transformers} capaz de realizar a predição de estruturas proteicas diretamente 
a partir de sequências de aminoácidos. 
Diferentemente de modelos como o \textit{AlphaFold},
que fazem uso extensivo de alinhamentos múltiplos de sequências (\textit{MSA, Multiple Sequence Alignments}) e atenção
baseada em grafos para inferir interações residuais,
o \textit{ESMFold} prioriza a velocidade de processamento, permitindo uma predição rápida de estruturas com precisão moderada.  
A arquitetura do \textit{ESMFold} baseia-se no modelo de linguagem proteica \textit{ESM-2}, 
um \textit{transformer} pré-treinado em grandes bancos de dados de sequências biológicas. 
Esse modelo aprende representações internas que capturam informações evolutivas e estruturais implícitas,
sem depender explicitamente de \textit{MSAs}. 
A predição estrutural ocorre por meio da inferência direta da topologia tridimensional da proteína a partir dessas representações,
minimizando a necessidade de computação intensiva e permitindo um tempo de inferência reduzido.
Embora o \textit{ESMFold} não alcance a precisão estrutural do \textit{AlphaFold},
sua abordagem é extremamente vantajosa para aplicações que exigem triagem de alto desempenho,
como análises em larga escala e estudos preliminares de modelagem proteica. 
Sua rapidez permite que estruturas sejam geradas em segundos, 
enquanto o \textit{AlphaFold} pode demandar minutos a horas,
dependendo da complexidade da proteína analisada. 

Ferramentas como o \textit{ProteinMPNN} \cite{ProteinMPNN} utilizam arquiteturas baseadas em \textit{Message Passing Neural Networks} 
(MPNNs) para prever sequências de aminoácidos capazes de adotar conformações tridimensionais específicas.
Diferentemente dos métodos tradicionais de \textit{design} de proteínas, 
que frequentemente utilizam abordagens estocásticas ou heurísticas para otimizar sequências, como o \textit{Rosetta},
o \textit{ProteinMPNN} emprega \textit{deep learning} para explorar de maneira eficiente o espaço de sequências
e gerar variantes estruturalmente viáveis.  
O modelo opera sobre a representação estrutural de uma proteína como um grafo,
onde os resíduos de aminoácidos correspondem a nós e as interações entre eles são representadas por arestas ponderadas. 
A passagem de mensagens ocorre entre os nós vizinhos, permitindo que o modelo aprenda padrões de interações residuais 
que favorecem a estabilidade estrutural da proteína. 
O modelo consiste de 3 camadas de \textit{encoder} e 3 de \textit{decoder} seguidos de uma camada escondida de 128 neurônios 
prevendo a sequência de aminoácidos de modo auto regressivo, do terminal N ao C. 



\cite{DyNAPPO} apresentaram o \textit{DyNA PPO}, que consiste em uma
abordagem de \textit{desing} de sequências proteicas baseada em aprendizado por reforço profundo (\textit{Deep Reinforcement Learning}, DRL),
atualizando seus parâmetros através do algorítimo \textit{Proximal Policy Optimization}, PPO.  
O \textit{DyNA PPO} modela o problema do design de proteínas como um processo de decisão de Markov, 
onde um agente gera sequências de aminoácidos de forma autoregressiva, 
adicionando resíduos um por um. 
A função de recompensa é baseada em métricas de estabilidade estrutural e funcionalidade da proteína, 
sendo calculada apenas ao final da geração da sequência. 
Esse mecanismo permite que o modelo aprenda estratégias eficazes para explorar o espaço de sequências de forma eficiente.  
O modelo é treinado iterativamente, ajustando sua política para maximizar a recompensa acumulada.
Durante o treinamento, um conjunto de redes neurais auxiliares avalia a qualidade das sequências geradas,
fornecendo sinais de aprendizado ao agente. 
Embora o \textit{DyNA PPO} tenha demonstrado resultados promissores em benchmarks computacionais, 
sua validação experimental ainda está em aberto. 

A análise da literatura existente evidencia que o design computacional de proteínas tem avançado significativamente, 
impulsionado por metodologias como modelagem molecular baseada em física, aprendizado profundo e aprendizado por reforço. 
Modelos como \textit{AlphaFold}, \textit{Rosetta}, \textit{ESMFold} e \textit{ProteinMPNN} têm contribuído de forma expressiva
para a predição estrutural e o \textit{design} de sequências proteicas.
Além disso, técnicas de aprendizado por reforço, como \textit{DyNA PPO}, 
demonstraram grande potencial para a otimização de sequências proteicas, 
explorando o espaço conformacional de maneira adaptativa e eficiente.  

No entanto, não foram encontrados estudos que apliquem aprendizado 
por reforço especificamente para a otimização do FIX no tratamento da hemofilia B. 
A pesquisa realizada não identificou publicações que abordem a aplicação dessa abordagem para modificar ou aprimorar a sequência do FIX,
de modo a aumentar sua estabilidade ou reduzir sua imunogenicidade como proposto pelo presente trabalho. 
