
\chapter{Revisão Bibliográfica}
\label{subsection:structure_pred}
%Explicar que, dado uma sequencia, preve qual é a estrutura


%Além do Rosetta, destacam-se o AlphaFold2 e o EMSFold como modelos relevantes para prever estruturas Z. L. e. al, 2022; Jumper, 2021. O primeiro utiliza MSAs (Alinhamento Múltiplo de Sequências) e templates de estruturas similares como entrada de uma rede neural profunda treinada para predizer a estrutura proteica Jumper, 2021. Entretanto, ao se processar uma predição, é necessário realizar uma busca em bancos de dados a fim de se obter os MSAs e os templates Jumper, 2021. Já o EMSFold utiliza um modelo de linguagem LLM (Large Language Model) batizado de ESM-2, composto por mais de 15 bilhões de parâmetros treinados para gerar uma representação da sequência que, por sua vez, serve de entrada para o ESMFold que prediz a disposição tridimensional dos átomos da estrutura Z. L. e. al, 2022. Por ter uma arquitetura que independe de buscas em bancos de dados, o EMSFold é capaz de realizar predições com uma ordem de grandeza mais rápidas que o AlphaFold2 e com acurácia comparável Z. L. e. al, 2022.


%Dar o argumento do De Novo que essas predições são confiaveis e generalizaveis para proteinas nao vistas. 
%De Novo
%Este trabalho teve o objetivo de verificar se os modelos
%deep learning treinados para prever estruturas, são capazes de fornecer informações relevantes o suficiente para criarmos proteinas novas, sem nenhuma relação com outras proteinas vistas na natureza. 
%A conclusão é que sim. 
%Para validar a hipotese, eles partiriam de uma sequencia aleatoria e geraramm o mapa de distancias inter-residuos usando o trRosetta. Com esse mapa eles criam uma distribuição de distâncias. A função objetivo é o KL entre essa distribuição e a distribuição média de todas as proteínas que foram utilizadas pra treinar o TrRosetta. 
%Ou seja, o trabalho não buscou chegar em alguma estrutura target, e sim numa distribuição de distancias inter-resíduos target. E isso fez com que as proteínas obtidas fossem estáveis, e totalmente novas. 
%Com essas proteinas, que são totalmente diferentes das proteinas vistas pelo modelo, eles produziram em laboratorio e confirmaram que a estrutura dessas proteinas estava de acordo com a estrututra prevista. 
%Ou seja, esses modelos de deep learning são capazes realmente de prever a estrutura, mesmo que seja uma proteina nunca antes vista. 
%O ponto é que existia a dúvida se esses modelos acertavam so as proteinas que existem e nao seriam capazes de generalizar proteinas nunca vistas. 
%Logo, esse trabalho traz como argumento que utilizar esses modelos para prever estruturas é fair, mesmo sendo pra prever estruturas que nao existem na natureza. 

%Se quiser falar desse trabalho aqui, precisaria explicar que ele não tem uma estrutura target e sim uma distribuição de distancias inter residuo target. 

%Acho melhor citar esse trabalho quando for falar de predicao de estruturas, usando de argumento que esses modelos são bons. 

%E procurar algum outro trabalho que maximize a similaridade entre estruturas. 

%----------------- REVISAR %-----------------%-----------------%-----------------
%A revisao bibliográfica foca nos trabalhos que foram feitos relacionados com o tema. No caso, design de proteina pra hemofilia B. 
%1- Começa com tratamentos atuais de hemofilia
%2- Abordagens de design de sequencias 
%Falar de Alphafold, ESMFold, Rosetta, ProteinMPNN, 



%O desenvolvimento de proteínas terapêuticas tem se beneficiado amplamente dos avanços na modelagem computacional e no aprendizado de máquina. Trabalhos pioneiros em bioengenharia molecular estabeleceram que a estrutura tridimensional de uma proteína determina sua função biológica e, portanto, o design de proteínas deve priorizar a geração de sequências que estabilizem conformações específicas de baixa energia livre. Métodos computacionais como o Rosetta \cite{Rosetta} são amplamente reconhecidos por sua capacidade de prever a estrutura de proteínas e otimizar interações moleculares com base em modelos de energia livre. Esses modelos consideram interações de Van der Waals, ligações de hidrogênio e solubilidade para minimizar a energia de dobramento e otimizar a funcionalidade.

%No contexto da hemofilia tipo B, variantes do fator IX (FIX) têm sido amplamente estudadas para melhorar a estabilidade e reduzir a frequência de infusões necessárias no tratamento. Por exemplo, a variante FIX-Padua apresentou maior atividade específica devido a uma mutação pontual (R338L), que aumenta sua eficiência catalítica \cite{FIXPadua}. Este avanço destaca o potencial do design de proteínas em superar limitações associadas ao FIX selvagem, como sua curta meia-vida e a necessidade de altas doses terapêuticas.

%Outra linha de pesquisa relevante foca na imunogenicidade de proteínas terapêuticas. Trabalhos como o de Peters et al. \cite{Peters2020} introduziram metodologias para prever epítopos de células T em proteínas, utilizando ferramentas como NetMHCIIpan. Essas ferramentas avaliam a afinidade de ligação entre peptídeos derivados de proteínas e moléculas do complexo de histocompatibilidade principal (MHC), um passo crucial na ativação de respostas imunológicas. No design de variantes do FIX, reduzir ou eliminar epítopos imunogênicos é fundamental para minimizar a formação de inibidores, uma das complicações mais graves em terapias baseadas em proteínas recombinantes.

%O uso de aprendizado de máquina no design de proteínas também tem se expandido rapidamente. Redes neurais profundas, como a ProteinMPNN \cite{ProteinMPNN}, permitem mapear estruturas tridimensionais para sequências de aminoácidos com base em características geométricas e químicas. Essas redes utilizam arquiteturas encoder-decoder para explorar o espaço de sequências de maneira eficiente, identificando combinações de aminoácidos que estabilizam estruturas desejadas. Trabalhos baseados em aprendizado por reforço, como AlphaFold e variantes de otimização estrutural, mostraram que algoritmos como o Proximal Policy Optimization (PPO) podem ser utilizados para gerar mutações direcionadas, otimizando métricas específicas, como a similaridade estrutural (TM-Score) ou a energia livre.

%Além disso, abordagens de otimização baseadas em busca iterativa, como o método de Monte Carlo implementado no Rosetta, são frequentemente comparadas com estratégias modernas baseadas em aprendizado de máquina. Enquanto o Rosetta depende de mutações aleatórias e avaliação direta de energia, modelos baseados em redes neurais e aprendizado por reforço introduzem mutações de forma mais direcionada, baseando-se em predições probabilísticas e dados pré-treinados.

%Em conjunto, esses trabalhos fornecem as bases conceituais e metodológicas para o desenvolvimento de novas variantes do FIX com propriedades otimizadas. A presente pesquisa combina ferramentas de aprendizado profundo, como o ProteinMPNN, e algoritmos de reforço, como PPO, para gerar sequências proteicas que mimetizem a estrutura do FIX com maior estabilidade, menor imunogenicidade e maior eficiência funcional. Este trabalho avança o estado da arte ao propor um pipeline integrado que incorpora metodologias modernas para superar as limitações dos métodos tradicionais de design de proteínas.

