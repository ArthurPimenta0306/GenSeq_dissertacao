%\chapter{Introdução}
\unnumberedchapter{Introdução}
\label{cap:introducao}
\enlargethispage{.5\baselineskip}

A hemofilia tipo B é uma doença hereditária rara, 
caracterizada por uma deficiência no Fator IX de coagulação humana (FIX). 
Os pacientes com a doença enfrentam um risco maior de sofrer com hemorragias graves, 
tanto internas quanto externas, 
podendo levar a complicações debilitantes e até mesmo à morte (\cite{Mannucci}).

Apesar dos avanços notáveis no tratamento da hemofilia nas últimas décadas,
muitos desafios ainda persistem.
O tratamento tradicional da hemofilia tipo B envolve a infusão de fatores de coagulação recombinantes
ou derivados do plasma sanguíneo (\cite{Gouw}). 
Embora eficazes na prevenção de hemorragias,
esses tratamentos apresentam limitações significativas,
como a necessidade de infusões frequentes devido à curta meia-vida do FIX na circulação,
a possibilidade de desenvolvimento de inibidores anticoagulantes e os elevados custos associados (\cite{Mancuso}).
Nesse contexto, o desenvolvimento de novas proteínas recombinantes (\textit{Sequence Design}) surge 
como uma abordagem com potencial de reduzir os custos associados ao tratamento da doença, 
projetando proteínas sob medida que desempenham a mesma função do FIX, mas com maior estabilidade,
reduzindo a frequência de infusões e o risco do desenvolvimento de inibidores

%\section{Sequence Design}
\unnumberedsection{Sequence Design}

\textit{Sequence Design} refere-se ao processo de criação ou
otimização de uma sequência de aminoácidos que se conformam em uma estrutura tridimensional de interesse.
Em outras palavras, consiste em desenvolver uma função $F_{seq}$ que 
recebe como entrada uma estrutura proteica alvo e retorna a sequência correspondente. 
Essa tarefa é desafiadora, 
pois o número de permutações possíveis para uma sequência de aminoácidos
cresce exponencialmente com seu tamanho, 
tornando inviável a busca dentre todas as possibilidades (\cite{Overview}).

\begin{figure}[H]
  \caption[\textit{Sequence Design}]{\textit{Sequence Design} se resume a desenvolver uma 
  função $F_{seq}$ que mapeia uma estrutura proteica tridimensional alvo em uma sequência de aminoácidos.}
  \centering
  \includegraphics[width=.8\textwidth]{figuras/metodologia-SeqDes.jpg}
\end{figure}

%\subsection{Sequence Design baseado em busca}
\unnumberedsubsection{Sequence Design baseado em busca}

No \textit{Sequence Design} baseado em busca, 
a função $F_{seq}$ é definida por um processo iterativo. 
Inicialmente, uma sequência de aminoácidos é gerada com base em uma heurística inicial $H$.
Essa sequência é então avaliada por uma função objetivo $F_{obj}$, 
responsável por calcular a métrica a ser otimizada.
A cada iteração, 
a nova sequência gerada é comparada com a sequência anterior de acordo com a métrica definida. 
Se a sequência atual apresentar melhor desempenho, o Agente $A$ a define como nova sequência final.
Caso contrário, a sequência final é mantida. 
Em ambos os cenários, o próximo passo é a aplicação de uma mutação pelo Agente $A$ sobre 
a sequência atual com o objetivo de explorar novas possibilidades no espaço de busca.
Esse processo é repetido até que um critério de parada seja atingido, 
como um número máximo de iterações ou a convergência da métrica calculada por $F_{obj}$.


\begin{figure}[H]
  \centering
  \includegraphics[width=.8\textwidth]{figuras/metodologia-SearchBased.jpg}
  \caption[\textit{Sequence Design} baseado em busca]{\textit{Sequence Design} baseado em busca define $F_{seq}$ 
           como um processo iterativo que combina uma 
           heurística inicial $H$, a avaliação por $F_{obj}$ e as modificações realizadas pelo Agente $A$. 
           O processo é repetido até atingir um número máximo de iterações ou um limiar calculado por $F_{obj}$.}
  \label{fig:seqdes_search_based}
\end{figure}

O software \textit{Rosetta} (\cite{Rosetta}) é uma das ferramentas 
mais amplamente utilizadas para otimização e \textit{design} de proteínas.
Ele implementa um \textit{pipeline} iterativo que combina heurísticas de geração de sequências iniciais, 
avaliações de estabilidade estrutural e técnicas de mutação orientada para explorar o espaço de sequências.
Em seu \textit{pipeline}:


\begin{itemize}
    \item \textbf{Heurística inicial ($H$):}  
    Utiliza a sequência de uma proteína estruturalmente semelhante como ponto de partida.

    \item \textbf{Função objetivo ($F_{obj}$):}  
    Avalia a energia livre da proteína resultante, considerando fatores como interações de Van der Waals,
    interações eletrostáticas e ligações de hidrogênio.

    \item \textbf{Agente ($A$):}  
    Utiliza o Método de Monte Carlo (MMC) para introduzir mutações na sequência e aceitar novas configurações com menor energia livre,
    guiando iterativamente o processo de otimização.
\end{itemize}

Abordagens alternativas podem usar a similaridade estrutural como função objetivo,
sendo uma métrica amplamente utilizada o \textit{Template Modeling Score} (TMScore) (\cite{tmscore}).

%\subsection{\textit{Sequence Design} baseado em Aprendizado Profundo}
\unnumberedsubsection{Sequence Design baseado em Aprendizado Profundo}
\textit{Sequence Design} baseado em aprendizado profundo define a função $F_{seq}$ 
a partir de redes neurais artificiais. 
O trabalho de \cite{ProteinMPNN} por exemplo, 
determina a $F_{seq}$ como uma rede neural profunda, denominada \textit{ProteinMPNN}, 
que mapeia de forma direta a estrutura alvo à sequência de aminoácidos. 
A rede é construída através de uma \textit{Message Passing Neural Network} (MPNN) 
composta por uma arquitetura \textit{encoder-decoder} 
que se baseia nas características da estrutura como distância e orientação dos átomos no espaço 
para fazer predições (\cite{ProteinMPNN}). 

\begin{figure}[H]
  \centering
  \includegraphics[width=.8\textwidth]{figuras/metodologia-DeepLearningBased.jpg}
  \caption{\textit{Sequence Design} baseado em aprendizado profundo.}
\end{figure}

%\section{\textit{Sequence Design} baseado em Aprendizado por Reforço Profundo} 
\unnumberedsection{\textit{Sequence Design} baseado em Aprendizado por Reforço Profundo}
\label{section:Proposta}

Além de suscitar uma possível resposta imunológica, 
o tratamento da hemofilia tipo B baseado na infusão de  FIX
é custoso devido ao pouco tempo de vida da proteína na corrente sanguínea, demandando infusões frequentes. 
Neste sentido, este trabalho busca responder à seguinte pergunta científica: 
É possível, através de \textit{Sequence Design}, projetar uma proteína que desempenhe a mesma função do FIX, 
mas com maior estabilidade, menor imunogenicidade e com necessidade de infusões reduzida? %menos necessidade de infusões?

Para responder a esta questão, formulamos a hipótese de que, 
se existir uma proteína alternativa superior ao FIX, 
ela deverá ser estruturalmente similar. 
Assim, o objetivo será identificar proteínas estruturalmente semelhantes que atendam aos critérios estabelecidos.
Para simplificar o problema, focaremos no projeto do domínio protease, 
reconhecido como o mais relevante para a função do FIX (\cite{FIX}).

Nossa dissertação consiste no desenvolvimento de um \textit{pipeline} que combina estratégias de busca 
com aprendizado profundo para otimização de sequências proteicas. 
Esse \textit{pipeline} será estruturado em três módulos principais: Condições Iniciais, 
Treinamento e Geração de Sequências.


O módulo de Condições Iniciais tem como objetivo obter a sequência inicial de aminoácidos 
por meio da heurística \( H \), além de calcular o erro inicial associado a essa sequência 
utilizando a função objetivo \( F_{\text{obj}} \). 
Para isso, empregaremos o \textit{ProteinMPNN} como heurística \( H \) 
e o TMScore como métrica de avaliação estrutural \( F_{\text{obj}} \).  

Diferente do uso de agentes que realizam mutações aleatórias, como no \textit{Rosetta},
no módulo de Treinamento uma rede neural profunda, o \textit{GenSeq}, será treinada para atuar como o Agente $A$,
sendo capaz de realizar mutações que otimizem a similaridade com a estrutura alvo, isto é, o FIX.
O \textit{GenSeq} será treinado utilizando um algoritmo de Aprendizado por Reforço Profundo (DRL), 
o \textit{Proximal Policy Optimization} (PPO) (\cite{PPO}).

Por fim, no módulo de Geração de Sequências, 
o agente treinado será utilizado para produzir um conjunto de variantes proteicas estruturalmente semelhantes ao FIX,
por meio de um processo de mutação orientado. 
As sequências geradas serão então avaliadas para identificar candidatas viáveis 
a substituir o FIX no tratamento da hemofilia tipo B.

\begin{figure}[H]
  \centering
  \includegraphics[width=.8\textwidth]{figuras/metodologia-pipeline_proposta.jpg}
  \caption{\textit{Pipeline} proposto de \textit{Sequence Design}.}
  \label{fig:proposta}
\end{figure}

%\section{Objetivos}
\unnumberedsection{Objetivos}

\begin{enumerate}
  \item Desenvolver um \textit{pipeline} genérico que produza sequências que mimetizem uma estrutura qualquer.
  \item Obter um conjunto de sequências de aminoácidos que mimetizem a estrutura do FIX baseado na métrica TMScore.
  \item Avaliar, dentro do conjunto de proteínas obtidas em 1, quais possuem potencial de substituir o FIX de modo tornar o tratamento de Hemofilia tipo B mais acessível e eficiente.
\end{enumerate}

%\section{Contribuições}
\unnumberedsection{Contribuições}
As contribuições deste trabalho são:
\begin{enumerate}
  \item O desenvolvimento de proteínas potenciais candidatas a substituir o FIX no tratamento da hemofilia tipo B.
  \item Um \textit{pipeline} para \textit{Sequence Design} baseado em Aprendizado por Reforço Profundo, que pode ser adaptado para outras proteínas com estruturas conhecidas.
  \item Introdução da métrica IMI (Imuno Índice), que quantifica a propensão de uma proteína gerar uma resposta imunológica adversa.
  \item O artefato do modelo de aprendizado por reforço profundo, o \textit{GenSeq}, que pode ser utilizado para geração de novas sequências proteicas a partir da estrutura do FIX (\cite{repo}).
  \item O código fonte das implementações e os scripts utilizados para o treinamento do \textit{GenSeq}, bem como a avaliação 
  de similaridade estrutural, \textit{docking} e imunogenicidade das sequências geradas (\cite{repo}).
  \item Os arquivos PDB utilizados como entrada para o treinamento do \textit{GenSeq} e para a avaliação das proteínas geradas (\cite{repo}).
  \item Apresentação dos resultados no \textit{World Federation of Hemophilia (WFH) 2024}, realizado em Madrid, Espanha, por meio do pôster científico intitulado \textit{“Protein Design and Deep Learning: A New Frontier in Hemophilia B Management”} (\cite{WFH2024})
\end{enumerate}