
\unnumberedchapter{Introdução}
{\color{red} Em andamento}


\label{cap:introducao}
\enlargethispage{.5\baselineskip}
A hemofilia tipo B é uma doença hereditária rara, 
caracterizada por uma deficiência no fator IX de coagulação sanguínea. 
Os pacientes com a doença enfrentam um risco maior de sofrer com hemorragias graves, 
tanto internas quanto externas, 
podendo levar a complicações debilitantes e até mesmo à morte \cite{Mannucci}.


Embora tenham sido obtidos avanços notáveis no tratamento da doença nas últimas décadas,
muitos desafios ainda persistem. 
O tratamento tradicional da hemofilia envolve a infusão de fatores de coagulação recombinantes 
ou derivados do plasma sanguíneo \cite{Gouw}. 
Apesar de sua eficácia na prevenção de hemorragias, 
esse tratamento possui limitações, 
tais como a necessidade de infusões frequentes, 
a possibilidade de desenvolvimento de inibidores anticoagulantes e, principalmente, 
os altos custos envolvidos \cite{Mancuso}. 
O projeto de sequências proteicas surge como uma abordagem com potencial de reduzir  
os custos associados ao tratamento da doença, 
ao projetar proteínas sob medida que desempenham a mesma função, porém, mais estáveis, 
exigindo uma quantidade menor de infusões. 


\section{Projeto de sequências}

O projeto de sequências se refere ao processo 
de criar ou otimizar uma sequência de aminoácidos de modo a produzir uma proteína 
cuja estrutura tridimensional possua características desejadas. 
Esta é uma tarefa desafiadora que, a depender do tamanho da sequência, 
se torna inviável de resolver através de uma busca por força bruta, 
devido às inúmeras permutações possíveis que se pode obter com os 20 aminoácidos 
conhecidos na natureza \cite{Overview}.  

\begin{figure}[H]
  \centering
  \includegraphics[width=.8\textwidth]{figuras/metodologia-SeqDes.jpg}
  \caption{O projeto de sequências se resume a desenvolver uma 
           função $F_{seq}$ que mapeia uma estrutura protéica tridimensional em uma sequência de aminoácidos. }
\end{figure}


\subsection{Baseado em busca}
O projeto de sequências baseado em busca, em geral, define a função $F_{seq}$ através do seguinte pipeline:

\begin{enumerate}
  \item Dada uma estrutura proteíca, é definida uma sequência de aminoácidos inicial, baseada em uma  heurística $H$.
  \item A sequência serve de entrada para a função objetivo $F_{obj}$ que calcula a métrica a ser otimizada pelo pipeline.
  \item Se a sequência atual é melhor do que sequência final, em termos da métrica calculada pela $F_{obj}$, então o Agente $A$ define a sequência atual como a nova sequência final. Além disso, $A$ também atualiza a sequência atual a partir de mutações.
  \item O processo se repete a partir do item 2 até que seja atingido um critério de parada. Geralmente é baseado no número de iterações e/ou na métrica objetivo. 
\end{enumerate}

\begin{figure}[H]
  \centering
  \includegraphics[width=.8\textwidth]{figuras/metodologia-SearchBased.jpg}
  \caption{Projeto de Sequências baseado em busca define $F_{seq}$ como um processo iterativo que se inicia em uma 
          heurística $H$, é avaliada por uma função $F_{obj}$ e modificada por um Agente $A$. O processo se repete até 
          atingir um limiar calculado por $F_{obj}$} 
  \label{fig:seqdes_search_based}
\end{figure}

O \textit{Rosetta} \cite{Rosetta} se baseia em um pipeline como este, 
onde a heurística $H$ define como partida uma sequência de uma proteína homóloga ou estruturalmente semelhante a estrutura alvo.  
Sua função objetivo $F_{obj}$ mede a energia livre da proteína,
considerando diversos fatores como interações de \textit{Van der Waals}, interações eletrostáticas, 
ligações de hidrogênio, solubilidade, entre outros. 
O \textit{Rosetta} busca minimizar a energia livre calculada pela $F_{obj}$. Portanto, seu Agente $A$ 
decide que a sequência final passa a ser a atual se houver um decréscimo na energia livre. Além disso, atribui modificações na sequência atual de forma aleatória, utilizando o Método de Monte Carlo (MMC).

{\color{red} citar um exemplo de abordagem que a função objetivo é a similaridade entre estruturas. }


\subsection{Baseado em aprendizado profundo}

O projeto de sequências baseado em aprendizado profundo define a função $F_{seq}$ 
a partir de redes neurais artificiais. 
O \cite{ProteinMPNN} por exemplo, 
determina a $F_{seq}$ como uma rede neural profunda, denominada \textit{ProteinMPNN}, 
que mapeia de forma direta a estrutura alvo à sequência de aminoácidos. 
A rede é construída através de uma \textit{Message Passing Neural Network} (MPNN) 
composta por uma arquitetura \textit{encoder-decoder} 
que se baseia nas características da estrutura como distância e orientação dos átomos no espaço, 
para fazer predições \cite{ProteinMPNN}. 

\begin{figure}[H]
  \centering
  \includegraphics[width=.8\textwidth]{figuras/metodologia-DeepLearningBased.jpg}
  \caption{Projeto de Sequências baseado em aprendizado profundo} 
\end{figure}


\section{Proposta} 
\label{section:Proposta}
%Problema: tratamento com Fator IX é caro por exigir inumeras infusoes pq o Fator IX é instavel. Alem disso, pode gerar resposta imune nos pacientes. 
%A pergunta cientifica é: É possível encontrar uma proteina melhor (mais estavel, mais barata de se produzir, menos resposta imune e capaz de exercer as mesmas funções da proteina original) que a proteina original utilizando um pipeline de busca, com RL, otimizando TMScore?
%A nossa hipótese é que, se houver uma proteína melhor que desempenha a mesma função, ela deve ter uma estrutura tridimensional similar. Logo, a busca será feita no espaço de proteínas similares. Logo, criamos um pipeline para gerar um conjunto de proteínas candidatas. 
Além de possívelmente provocar uma resposta imune no paciente, 
o tratamento da hemofilia tipo B, baseado na infusão do Fator IX, 
é custoso devido ao pouco tempo de vida do Fator IX na corrente sanguínea, demandando infusões frequentes. 
Neste sentido, o presente trabalho se propõe a responder a seguinte pergunta científica: 
É possível, através de projeto de sequências, projetar uma proteína que desempenhe a mesma função do Fator IX mas com 
um perfil imunológico melhor e que demande menos infusões no tratamento da hemofilia tipo B?

Para responder a esta questão, formulamos a hipótese de que, 
se existir uma proteína alternativa superior ao Fator IX, 
ela deverá ser estruturalmente similar. 
Com base nisso, pretendemos selecionar, dentre um conjunto de proteínas estruturalmente semelhantes ao Fator IX,
aquelas que satisfazem os critérios estabelecidos, caso existam.

{\color{red} Explicar que, para simplificar o problema, vamos projetar apenas um dominio, a protease, que é o mais relevante}

De modo a obter esse conjunto,
será desenvolvido um pipeline que combina a arquitetura do projeto de sequências baseado em busca e técnicas de aprendizado profundo
para gerar novas proteínas com estruturas similares ao do FIX.
O processo será dividido em três módulos: Condições Iniciais, Treinamento e Geração de Sequências. 

O módulo de Condições iniciais consiste na obtenção da sequência inicial de aminoácidos através da heurística $H$,
bem como o cálculo do erro inicial associado a esta sequência, usando a função objetivo $F_{obj}$. 
Vamos utilizar o \textit{ProteinMPNN} como heurística $H$ e a métrica de similaridade estrutural 
\textit{Template Matching Score} (\textit{TMScore}) como função objetivo.

Diferente do uso de agentes que realizam mutações aleatórias, como no \textit{Rosetta},
no módulo de Treinamento, treinaremos uma rede neural profunda, o \textit{GenSeq}, para atuar como o Agente $A$,
sendo capaz de realizar mutações que otimizem a similaridade com a estrutura alvo, isto é, o Fator IX.

Após o treinamento, no módulo de Geração de Sequências, 
espera-se que o Agente seja capaz de gerar várias proteínas estruturalmente semelhantes à estrutura alvo por meio de mutações direcionadas. 
A partir das sequências geradas, avaliaremos a existência de candidatas viáveis para substituir o Fator IX no tratamento da hemofilia tipo B.



\begin{figure}[H]
  \centering
  \includegraphics[width=.8\textwidth]{figuras/metodologia-pipeline_proposta.jpg}
  \begin{itemize}
  \item o \textit{ProteinMPNN} será utilizado como heurística $H$. 
  \item Na função objetivo $F_{obj}$ será calculada uma métrica de similaridade entre a estrutura da proteína atual 
  e a estrutura alvo que pretendemos maximizar: \textit{Template Matching Score} - \textit{TMScore}.
  \item O Agente $A$ será uma rede neural profunda, nomeada de  \textit{GenSeq},
  treinada para receber uma sequência de aminoácidos e prever a mutação que maximiza a similaridade com a estrutura alvo, i.e, otimiza $F_{obj}$. 
  O treinamento do $A$ será feito com aprendizado por reforço profundo, utilizando a ténica \textit{Proximal Policy Optimization} - PPO
  \item Depois de treinado, $A$ será utilizado como gerador de proteínas estruturalmente semelhantes ao Fator IX. 
\end{itemize}
  \caption{Projeto de Sequências proposto} 
  \label{fig:proposta}
\end{figure}

\section{Objetivos} 

Definimos os seguintes objetivos:

\begin{enumerate}
  \item Obter um conjunto de sequências de aminoácidos que mimetizem a estrutura do fator de coagulação IX baseado na métrica TM-Score.
  \item Avaliar, dentro do conjunto de proteínas obtidas em 1, quais possuem potencial de substituir o Fator IX de modo tornar o tratamento de Hemofilia tipo B mais acessível e eficiente.
  \item Desenvolver um \textit{pipeline} genérico que produza sequências que mimetizem uma estrutura qualquer.
\end{enumerate}






 
