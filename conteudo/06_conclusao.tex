\chapter{Conclusão}

Este trabalho apresentou um novo paradigma para o projeto de proteínas terapêuticas,
aplicando aprendizado por reforço profundo para otimizar a sequência do fator IX (FIX),
com o objetivo de melhorar sua estabilidade estrutural, 
reduzir a imunogenicidade e potencializar sua eficácia no tratamento da hemofilia tipo B.  

A abordagem proposta combinou metodologias avançadas de modelagem molecular e inteligência artificial 
para explorar de maneira eficiente o vasto espaço de sequências proteicas. 
O uso do algoritmo \textit{Proximal Policy Optimization} (PPO) permitiu a evolução de variantes do FIX 
por meio de um processo iterativo de seleção e refinamento, 
guiado por métricas estruturais. 
A validação das proteínas geradas foi realizada por meio de análises computacionais, 
incluindo cálculos de similaridade estrutural com o TMScore, identificação da presença de epítopos imunogênicos nas sequências geradas
e avaliação da interação proteína-proteína através do \textit{docking} molecular com o Fator VIII.
Os resultados indicaram que algumas variantes apresentam potencial terapêutico promissor.  

A principal contribuição deste trabalho reside na aplicação inédita de aprendizado por reforço profundo
para a otimização do FIX, um campo ainda pouco explorado na literatura científica. 
A inexistência de estudos anteriores que utilizem essa abordagem para o projeto de proteínas 
voltadas ao tratamento da hemofilia B reforça a originalidade e a relevância deste estudo.  

Embora os resultados obtidos sejam promissores, 
é importante ressaltar que ainda é necessário validar experimentalmente (\textit{wet lab})
as proteínas geradas, a fim de confirmar sua estabilidade e atividade biológica.
Além disso, é fundamental a realização de ensaios pré-clínicos e clínicos
para avaliar a segurança e eficácia dessas proteínas em pacientes com hemofilia B.

Em síntese, este trabalho demonstra o potencial do aprendizado por reforço profundo no projeto 
de proteínas terapêuticas. 
Os avanços obtidos podem servir de base para novas pesquisas que busquem aprimorar o 
desenvolvimento de biofármacos personalizados, 
contribuindo para a inovação no tratamento da hemofilia tipo B e outras doenças genéticas.

