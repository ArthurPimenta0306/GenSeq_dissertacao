\chapter{Conclusão}

Este trabalho apresentou um novo paradigma para o desenvolvimento de proteínas terapêuticas,
aplicando aprendizado por reforço profundo para otimizar a sequência do fator IX (FIX),
com o objetivo de melhorar sua estabilidade estrutural, 
reduzir a imunogenicidade e potencializar sua eficácia no tratamento da hemofilia tipo B.  


A abordagem proposta combinou metodologias avançadas de modelagem molecular e inteligência artificial 
para explorar de maneira eficiente o vasto espaço de sequências proteicas. 
Utilizando o algoritmo \textit{Proximal Policy Optimization} (PPO)
desenvolvemos o \textit{GenSeq}, um modelo capaz de realizar mutações guiadas em sequências proteicas,
permitindo a evolução de variantes do FIX por meio de um processo iterativo
guiado por métricas estruturais.
A validação das proteínas geradas foi realizada por meio de análises computacionais, 
incluindo cálculos de similaridade estrutural com o \textit{TMScore}, 
avaliação da interação proteína-proteína através do \textit{docking} molecular com o Fator VIII
e identificação da presença de epítopos imunogênicos nas sequências geradas.
Os resultados indicaram que algumas variantes apresentam potencial terapêutico promissor.
Em relação aos objetivos estabelecidos, todos foram atendidos:

\begin{enumerate}
    \item \textbf{Obter um conjunto de sequências de aminoácidos que mimetizem a estrutura do FIX baseado na métrica TMScore:} 
    Foram geradas diversas variantes com alta similaridade estrutural ao FIX nativo, sendo 
    63 delas com \textit{TMScore} superior a 92\%,
    validando a capacidade do \textit{pipeline} em preservar a conformação tridimensional desejada.
    \item \textbf{Avaliar, dentro do conjunto de proteínas obtidas em 1, quais possuem potencial de substituir o FIX de modo tornar o tratamento de Hemofilia tipo B mais acessível e eficiente:} 
    As análises computacionais indicaram que várias das proteínas geradas apresentaram 
    maior estabilidade de interação com o Fator VIII (FVIII) 
    e menor risco imunogênico segundo a nova métrica introduzida, 
    o Índice de Compatibilidade Imunológica (IMI). 
    \item \textbf{Desenvolver um \textit{pipeline} genérico que produza sequências que mimetizem uma estrutura qualquer:} 
    O \textit{pipeline} possui arquitetura modular, 
    permitindo a parametrização da estrutura alvo, 
    da função de recompensa e dos critérios de avaliação. 
    Dessa forma, ele pode ser aplicado a outros casos de interesse 
    em bioengenharia e no desenvolvimento de proteínas terapêuticas personalizadas.
\end{enumerate}


Além disso, neste trabalho foi introduzido o IMI (Imuno Índice),
uma métrica desenvolvida para quantificar a compatibilidade imunológica de proteínas projetadas 
em relação ao perfil imune da população humana. 
O IMI mede o desalinhamento entre os epítopos 
preditos em uma proteína e a frequência dos alelos na população, 
permitindo avaliar a propensão de uma resposta imunológica adversa.
Esta métrica foi fundamental para comparar as variantes geradas e 
identificar quais proteínas apresentavam menor risco de desencadear uma resposta imunológica.


A resposta à pergunta científica que este trabalho se propôs a responder é positiva. 
É possível, por meio de \textit{Sequence Design} com aprendizado por reforço profundo, 
projetar variantes do FIX com propriedades otimizadas para uso terapêutico. 
Ainda que os resultados sejam preliminares e necessitem de validação experimental 
em laboratório (wet lab), 
os dados computacionais apontam para um caminho viável 
na geração de proteínas terapêuticas personalizadas e com menor risco imunológico.


