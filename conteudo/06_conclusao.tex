\chapter{Conclusão}
{\color{red} Em andamento}


Este trabalho apresentou um novo paradigma para o \textit{design} de proteínas terapêuticas,
aplicando aprendizado por reforço profundo para otimizar a sequência do fator IX (FIX),
com o objetivo de melhorar sua estabilidade estrutural, 
reduzir a imunogenicidade e potencializar sua eficácia no tratamento da hemofilia tipo B.  

A abordagem proposta combinou metodologias avançadas de modelagem molecular e inteligência artificial 
para explorar de maneira eficiente o vasto espaço de sequências proteicas. 
O uso do algoritmo \textit{Proximal Policy Optimization} (PPO) permitiu a evolução de variantes do FIX 
por meio de um processo iterativo de seleção e refinamento, 
guiado por métricas estruturais e funcionais. 
A validação das proteínas geradas foi realizada por meio de análises computacionais, 
incluindo \textit{Template Modeling Score} (TM-Score), 
\textit{Contact Molecular Surface} (CMS) e \textit{docking} molecular com o Fator VIII (FVIII), 
demonstrando que algumas variantes apresentam potencial terapêutico promissor.  

A principal contribuição deste trabalho reside na aplicação inédita de aprendizado por reforço profundo
para a otimização do FIX, um campo ainda pouco explorado na literatura científica. 
A inexistência de estudos anteriores que utilizem essa abordagem para o design de proteínas 
voltadas ao tratamento da hemofilia B reforça a originalidade e a relevância deste estudo.  

Embora os resultados obtidos sejam promissores, 
este trabalho apresenta algumas limitações. 
A ausência de validação experimental (\textit{wet lab}) impede a confirmação direta 
da funcionalidade das proteínas geradas, 
sendo necessário realizar testes laboratoriais para avaliar sua estabilidade e atividade biológica. 
Além disso, futuras melhorias podem incluir a incorporação de simulações físicas
mais sofisticadas para refinar as predições estruturais e funcionais do modelo.  

Em síntese, este trabalho demonstra o potencial do aprendizado por reforço profundo no design racional
de proteínas terapêuticas. 
Os avanços obtidos aqui podem servir de base para novas pesquisas que busquem aprimorar o 
desenvolvimento de biofármacos personalizados, 
contribuindo para a inovação no tratamento da hemofilia tipo B e outras doenças genéticas.

