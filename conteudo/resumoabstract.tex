%!TeX root=../tese.tex
%("dica" para o editor de texto: este arquivo é parte de um documento maior)
% para saber mais: https://tex.stackexchange.com/q/78101

% As palavras-chave são obrigatórias, em português e em inglês, e devem ser
% definidas antes do resumo/abstract. Acrescente quantas forem necessárias.


%\palavraschave{Palavra-chave1, Palavra-chave2, Palavra-chave3}

%\keywords{Keyword1,Keyword2,Keyword3}

% O resumo é obrigatório, em português e inglês. Estes comandos também
% geram automaticamente a referência para o próprio documento, conforme
% as normas sugeridas da USP.




%\palavraschave{Hemofilia B, Fator IX, Design de proteínas, Aprendizado por Reforço, PPO, Proximal Policy Optimization, Imunogenicidade, Docking Molecular}

%\keywords{Hemophilia B, Factor IX, Protein Design, Reinforcement Learning, PPO, Immunogenicity, Molecular Docking}
\palavrachave{Hemofilia B, Fator IX, Design de proteínas, Aprendizado por Reforço, PPO, Imunogenicidade, Docking Molecular}
\keyword{Hemophilia B, Factor IX, Protein Design, Reinforcement Learning, PPO, Immunogenicity, Molecular Docking}

\resumo{
    Neste trabalho, propomos o desenvolvimento de um pipeline computacional
    para o \textit{design} de proteínas substitutas ao Fator IX de coagulação humana (FIX), 
    com o objetivo de otimizar seu uso terapêutico no tratamento da hemofilia B. 
    A metodologia combina aprendizado por reforço profundo com técnicas avançadas de modelagem estrutural 
    e análise de interação molecular. 
    O pipeline é dividido em três módulos principais: (i) Condições Iniciais, 
    responsável por gerar a sequência inicial de aminoácidos 
    e avaliar sua estrutura; 
    (ii) Trereforço 
    para realizar mutações na sequência inicial que minamento, no qual um agente \textit{GenSeq} é treinado com aprendizado por aximizem a similaridade estrutural com o FIX; 
    e (iii) Geração de Sequências, que utiliza o agente treinado para explorar o espaço de sequências. 
    A avaliação das proteínas geradas inclui a análise de similaridade estrutural, 
    o estudo das interações moleculares com outras macromoléculas, como o Fator VIII de coagulação humana (FVIII),
    e a avaliação do perfil imunológico. 
    Os resultados demonstram que as proteínas propostas apresentam alta similaridade estrutural com o FIX nativo, 
    interações moleculares eficientes com o FVIII e perfis imunogênicos comparáveis ou mais favoráveis que o FIX.
    O pipeline mostrou-se promissor como ferramenta para o \textit{design} de proteínas terapêuticas personalizadas 
    e de baixo risco imunológico para aplicação clínica no tratamento da hemofilia B.
}

\abstract{
    This work proposes the development of a computational pipeline for 
    designing substitute proteins for Human coagulation Factor IX (FIX),
    aiming to optimize their therapeutic application in the treatment of hemophilia B. 
    The methodology combines deep reinforcement learning with advanced structural modeling 
    and molecular interaction analysis techniques. 
    The pipeline is divided into three main modules: (i) Initial Conditions, 
    responsible for generating the initial amino acid sequence  
    and predicting its structure; 
    (ii) Training, where an agent named \textit{GenSeq} is trained by 
    deep reinforcement learning to perform mutations that maximize structural similarity to FIX; 
    and (iii) Sequence Generation, where the trained agent explores the space of sequences. 
    The evaluation of the generated proteins includes structural similarity analysis,
    molecular docking studies with Human coagulation Factor FVIII (FVIII),
    and the immunogenic profile evaluation. 
    The results demonstrate that the designed proteins present high structural similarity to the native FIX, 
    efficient molecular interactions with FVIII, and immunogenicity profiles comparable to or lower than the reference FIX.
    The proposed pipeline proved to be a promising tool for the design of personalized therapeutic proteins 
    with reduced immunogenic risk for clinical application in the treatment of hemophilia B.
}



    