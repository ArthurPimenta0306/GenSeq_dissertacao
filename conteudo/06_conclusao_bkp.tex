\chapter{Conclusão}

Este trabalho apresentou um novo paradigma para o desenvolvimento de proteínas terapêuticas,
aplicando aprendizado por reforço profundo para otimizar a sequência do fator IX (FIX),
com o objetivo de melhorar sua estabilidade estrutural, 
reduzir a imunogenicidade e potencializar sua eficácia no tratamento da hemofilia tipo B.  


A abordagem proposta combinou metodologias avançadas de modelagem molecular e inteligência artificial 
para explorar de maneira eficiente o vasto espaço de sequências proteicas. 
Utilizando o algoritmo \textit{Proximal Policy Optimization} (PPO)
desenvolvemos o \textit{GenSeq}, um modelo capaz de realizar mutações guiadas em sequências proteicas,
permitindo a evolução de variantes do FIX por meio de um processo iterativo
guiado por métricas estruturais.
A validação das proteínas geradas foi realizada por meio de análises computacionais, 
incluindo cálculos de similaridade estrutural com o \textit{TMScore}, 
avaliação da interação proteína-proteína através do \textit{docking} molecular com o Fator VIII
e identificação da presença de epítopos imunogênicos nas sequências geradas.
Os resultados indicaram que algumas variantes apresentam potencial terapêutico promissor. 

É importante destacar que o \textit{pipeline} desenvolvido é genérico, 
podendo ser adaptado para otimizar outras proteínas terapêuticas que tenham suas estruturas tridimensionais
conhecidas. A arquitetura modular do \textit{pipeline} facilita a substituição dos modelos de entrada,
das funções objetivo e dos critérios de avaliação.

Além disso, o trabalho introduziu o IMI (Imuno Índice),
uma métrica desenvolvida para quantificar a compatibilidade imunológica de proteínas projetadas 
em relação ao perfil imune da população humana. 
O IMI mede o desalinhamento entre os epítopos 
preditos em uma proteína e a frequência dos alelos na população, 
permitindo avaliar a propensão de uma resposta imunológica adversa.
Esta métrica foi fundamental para comparar as variantes geradas e 
identificar quais proteínas apresentavam menor risco de desencadear uma resposta imunológica.

Embora os resultados obtidos sejam promissores, 
é importante ressaltar que ainda é necessário validar experimentalmente (\textit{wet lab})
as proteínas geradas, a fim de confirmar sua estabilidade e atividade biológica.
Além disso, é fundamental a realização de ensaios pré-clínicos e clínicos
para avaliar a segurança e eficácia dessas proteínas em pacientes com hemofilia B.

Em síntese, este trabalho demonstra o potencial do aprendizado por reforço profundo no desenvolvimento 
de proteínas terapêuticas. 
Os avanços obtidos podem servir de base para novas pesquisas que busquem aprimorar a 
geração de biofármacos personalizados, 
contribuindo para a inovação no tratamento da hemofilia tipo B e outras doenças genéticas.

